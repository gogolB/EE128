\documentclass{article}

\usepackage[margin=1in]{geometry}
\usepackage{listings}
\usepackage{color}
\usepackage{graphicx}
\usepackage{float}

\definecolor{dkgreen}{rgb}{0,0.6,0}
\definecolor{gray}{rgb}{0.5,0.5,0.5}
\definecolor{mauve}{rgb}{0.58,0,0.82}

\lstset{frame=tb,
	language=C,
	aboveskip=3mm,
	belowskip=3mm,
	showstringspaces=false,
	columns=flexible,
	basicstyle={\small\ttfamily},
	numbers=left,
	numberstyle=\tiny\color{gray},
	keywordstyle=\color{blue},
	commentstyle=\color{dkgreen},
	stringstyle=\color{mauve},
	breaklines=true,
	breakatwhitespace=true,
	tabsize=4
}

\title{HW 2: Serial Communication, Optimization, and Feedback Control}
\date{November 27, 2017}
\author{Souradeep Bhattacharya \\ EE128 Section: 001}
\begin{document}
	\maketitle
	\section*{Question 1: SCI}
	\subsection*{Part A}
	The advantages of serial over parallel communications are as follows:
	\begin{itemize}
		\item Uses less wires
		\item It less expensive to implement
		\item It is easier to increase the the signal frequency.
	\end{itemize}
	\subsection*{Part B}
	\begin{lstlisting}
	void initSCI1()
	{
		// Clock at 14400
		SC1BD = 104;
		
		// Enable Odd Parity and Parity Checking
		SCI1CR1 = 0x03;
		
		// Enable TX, RX, RXFI, TXEI
		SCI1CR2 = 0x6c;
	}
	\end{lstlisting}
	\section*{Question 2: SPI}
	\begin{lstlisting}
	void initSPI()
	{
		SPI0BR = 0x52;
		SPI0CR1 = 0xFD;
	}
	\end{lstlisting}
	\section*{Question 3: I2C}
	\subsection*{Part A}
	The address is 0b0011010 or 0x1A
	\subsection*{Part B}
	Data is being written to slave device. Data is 0b00010110 or 0x12.
	\section*{Question 4: PID Control}
	\begin{lstlisting}
	// Time constants. Will be initialized in main().
	double Kp; // proportional gain
	double Ki; // integral gain
	double Kd; // derivative gain
	
	double cumulative_error = 0; // stores the sum of all prior errors
	double previous_output = 0; // stores the value of prior output
	
	// make_decision(): generates a drive signal
	// - setpoint: desired condition
	// - current_output: current condition measured by sensors
	double make_decision(double setpoint, double current_output) {
	
		// PID terms
		double P, I, D;
		
		// Question (a)
		P = Kp * (setpoint - current_output);
		
		// Question (b)
		cumulative_error += (setpoint - current_output);
		I = Ki * cumulative_error;
		
		// Question (c)
		D = Kd * (current_output - previous_output);
		previous_output = current_output;
		
		return u = P + I - D;
	}
	\end{lstlisting}
\end{document}